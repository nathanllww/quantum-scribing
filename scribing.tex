\documentclass{article}

\usepackage{amsmath, amssymb, amsthm, braket}
\usepackage[margin=1in]{geometry}
%\usepackage[parfill]{parskip}

% \usepackage{titlesec}
% \titleformat{\section}{
% \vspace{-4pt}\large\bf
% }{}{0em}{}[\vspace{-5pt}]

\newtheorem{theorem}{Theorem}
\newtheorem{lemma}[theorem]{Lemma}
\newtheorem{corollary}[theorem]{Corollary}
\newtheorem{proposition}[theorem]{Proposition}
\newtheorem{definition}[theorem]{Definition}
\newtheorem{example}[theorem]{Example}
\newtheorem{claim}[theorem]{Claim}

\title{Lecture 6 Scribing Notes}
\author{Dishen Wang, Nathan White, and Han Cao \\ CS 496: Foundation of Quantum Computing}
\date{\today}

\begin{document}
\maketitle
\section{Partial Measurements}
This lecture begins with a recap and expansion of the concept of \textit{partial measurements}.
Consider some state over two qubits; any such state can be written as
\[\alpha_{00}\ket{00} + \alpha_{01}\ket{01} + \alpha_{10}\ket{10} + \alpha_{11}\ket{11}\]
where $|\alpha_{00}|^2+|\alpha_{01}|^2+|\alpha_{10}|^2+|\alpha_{11}|^2=1$.
Suppose Alice measures her (the first) qubit.
Then,
\begin{itemize}
    \item Her qubit collapses to 0, which has probability $p_0 = |\alpha_{00}|^2+|\alpha_{01}|^2$.
        In this case, the state becomes $(1/\sqrt{p_0})(\alpha_{00}\ket{00}+\alpha_{01}\ket{01}) = (1/\sqrt{p_0})\ket{0}\otimes (\alpha_{00}\ket{0}+\alpha_{01}\ket{1})$.
        This is no longer entangled, since Alice's qubit has been measured, but the probabilities remain the same as in the complete measurement.
        Bob's probability of seeing 0 after Alice sees 0 is $|\alpha_{00}|^2/p_0$, and since Alice see 0 with probability $p_0$, they both see 0 with probability $|\alpha_{00}|^2$, as expected.
    \item Her qubit collapses to 1, which has probability $p_1 = 1-p_0$.
        By symmetry, the analogous situation to collapsing to 0 occurs.
\end{itemize}
From this exercise, we see that Alice measuring and Bob later measuring is the same as measuring the entire state.
Moreover, the exact same argument follows with Bob measuring first; thus order of measuring does not matter.
\end{document}
